\documentclass[a4paper]{article}
\usepackage[portuges]{babel}
\usepackage[latin1]{inputenc}
\usepackage{url}
\usepackage{listings}

\setlength{\oddsidemargin}{-1cm}
\setlength{\textwidth}{18cm}
\setlength{\headsep}{-1cm}
\setlength{\textheight}{23cm}

\parindent=0pt
\parskip=2pt
\newtheorem{questao}{Quest\~{a}o}

\def\wiki{\textsf{Wikipedia}}

\begin{document}

\title{}
\author{Trabalho Prático nº 1\\ (Lex)}
\date{Ano lectivo 10/11}

\maketitle

\section{Análise do problema}
\label{sec:analise}
%--------------------------------------------------------------------------
\section{Geração de ficheiros e Estruturas de dados}
A análise do problema apresentado em \ref{sec:analise} conclui a segmentação da resolução em dois módulos fundamentais. O componente lex permite a análise léxica e produção de tokens a serem consumidos pelo componente bibBase. Este último suporta a informação  contida no documento original num formato por nós escolhido de forma a permitir a criação dos ficheiros html e dot conforme requerido no enunciado.

Os tokens a produzir pelo lex são descritos conforme a estrutura presente em \ref{fig:token}. Estes tokens suportam  parte da informação contida numa citação qualquer presente no ficheiro bibTex. Além destes o bibBase  precissa de saber a  categoria correspondente à chave de citação. Uma adição à base de conhecimento de uma citação necessita \textbf{obrigatoriamente} destes componentes.


A estrutura de suporte - que contém toda a informação produzida pelo lex - é uma hashTable indexada pela categoria em que para cada entrada desta se podem encontrar uma lista ligada de referencias (\ref{fig:token}). Sendo a produção do ficheiro html facilitada e complicado um pouco a do dot em termos de complexidade. A estrutura pode-se ver na figura (\ref{fig:estrutura}).  A partir desta pode-se facilmente proceder à geração dos ficheiros indicados percorrendo a estrutura e gerando o ficheiro ao mesmo tempo. 

O caso do dot introduz complexidade. A geração do documento é um bocado pesada visto que é necessário percorrer todas as citações presentes na estrutura. A complexidade da produção do documento não foi considerada por nós . O problema seria facilmente resolvido introduzindo um maior gasto de memória pela nossa parte, mas não tido sido considerado por nós como objectivo do trabalho , não o fizémos. 


\end{document}
